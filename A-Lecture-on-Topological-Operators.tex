% Options for packages loaded elsewhere
\PassOptionsToPackage{unicode}{hyperref}
\PassOptionsToPackage{hyphens}{url}
\PassOptionsToPackage{dvipsnames,svgnames,x11names}{xcolor}
%
\documentclass[
  letterpaper,
  DIV=11,
  numbers=noendperiod]{scrreport}

\usepackage{amsmath,amssymb}
\usepackage{iftex}
\ifPDFTeX
  \usepackage[T1]{fontenc}
  \usepackage[utf8]{inputenc}
  \usepackage{textcomp} % provide euro and other symbols
\else % if luatex or xetex
  \usepackage{unicode-math}
  \defaultfontfeatures{Scale=MatchLowercase}
  \defaultfontfeatures[\rmfamily]{Ligatures=TeX,Scale=1}
\fi
\usepackage{lmodern}
\ifPDFTeX\else  
    % xetex/luatex font selection
\fi
% Use upquote if available, for straight quotes in verbatim environments
\IfFileExists{upquote.sty}{\usepackage{upquote}}{}
\IfFileExists{microtype.sty}{% use microtype if available
  \usepackage[]{microtype}
  \UseMicrotypeSet[protrusion]{basicmath} % disable protrusion for tt fonts
}{}
\makeatletter
\@ifundefined{KOMAClassName}{% if non-KOMA class
  \IfFileExists{parskip.sty}{%
    \usepackage{parskip}
  }{% else
    \setlength{\parindent}{0pt}
    \setlength{\parskip}{6pt plus 2pt minus 1pt}}
}{% if KOMA class
  \KOMAoptions{parskip=half}}
\makeatother
\usepackage{xcolor}
\setlength{\emergencystretch}{3em} % prevent overfull lines
\setcounter{secnumdepth}{5}
% Make \paragraph and \subparagraph free-standing
\ifx\paragraph\undefined\else
  \let\oldparagraph\paragraph
  \renewcommand{\paragraph}[1]{\oldparagraph{#1}\mbox{}}
\fi
\ifx\subparagraph\undefined\else
  \let\oldsubparagraph\subparagraph
  \renewcommand{\subparagraph}[1]{\oldsubparagraph{#1}\mbox{}}
\fi


\providecommand{\tightlist}{%
  \setlength{\itemsep}{0pt}\setlength{\parskip}{0pt}}\usepackage{longtable,booktabs,array}
\usepackage{calc} % for calculating minipage widths
% Correct order of tables after \paragraph or \subparagraph
\usepackage{etoolbox}
\makeatletter
\patchcmd\longtable{\par}{\if@noskipsec\mbox{}\fi\par}{}{}
\makeatother
% Allow footnotes in longtable head/foot
\IfFileExists{footnotehyper.sty}{\usepackage{footnotehyper}}{\usepackage{footnote}}
\makesavenoteenv{longtable}
\usepackage{graphicx}
\makeatletter
\def\maxwidth{\ifdim\Gin@nat@width>\linewidth\linewidth\else\Gin@nat@width\fi}
\def\maxheight{\ifdim\Gin@nat@height>\textheight\textheight\else\Gin@nat@height\fi}
\makeatother
% Scale images if necessary, so that they will not overflow the page
% margins by default, and it is still possible to overwrite the defaults
% using explicit options in \includegraphics[width, height, ...]{}
\setkeys{Gin}{width=\maxwidth,height=\maxheight,keepaspectratio}
% Set default figure placement to htbp
\makeatletter
\def\fps@figure{htbp}
\makeatother

%%%% header.tex

%packages
\usepackage{mathtools}

%macros
\DeclareMathOperator{\vol}{vol}
\DeclareMathOperator{\U}{U}
\DeclareMathOperator{\SU}{SU}
\DeclareMathOperator{\imunit}{i}
\DeclareMathOperator{\id}{id}
\DeclareMathOperator{\Map}{Map}
\newcommand{\stdim}{D}


%%%% end header.tex
\KOMAoption{captions}{tableheading}
\makeatletter
\@ifpackageloaded{tcolorbox}{}{\usepackage[skins,breakable]{tcolorbox}}
\@ifpackageloaded{fontawesome5}{}{\usepackage{fontawesome5}}
\definecolor{quarto-callout-color}{HTML}{909090}
\definecolor{quarto-callout-note-color}{HTML}{0758E5}
\definecolor{quarto-callout-important-color}{HTML}{CC1914}
\definecolor{quarto-callout-warning-color}{HTML}{EB9113}
\definecolor{quarto-callout-tip-color}{HTML}{00A047}
\definecolor{quarto-callout-caution-color}{HTML}{FC5300}
\definecolor{quarto-callout-color-frame}{HTML}{acacac}
\definecolor{quarto-callout-note-color-frame}{HTML}{4582ec}
\definecolor{quarto-callout-important-color-frame}{HTML}{d9534f}
\definecolor{quarto-callout-warning-color-frame}{HTML}{f0ad4e}
\definecolor{quarto-callout-tip-color-frame}{HTML}{02b875}
\definecolor{quarto-callout-caution-color-frame}{HTML}{fd7e14}
\makeatother
\makeatletter
\@ifpackageloaded{bookmark}{}{\usepackage{bookmark}}
\makeatother
\makeatletter
\@ifpackageloaded{caption}{}{\usepackage{caption}}
\AtBeginDocument{%
\ifdefined\contentsname
  \renewcommand*\contentsname{Table of contents}
\else
  \newcommand\contentsname{Table of contents}
\fi
\ifdefined\listfigurename
  \renewcommand*\listfigurename{List of Figures}
\else
  \newcommand\listfigurename{List of Figures}
\fi
\ifdefined\listtablename
  \renewcommand*\listtablename{List of Tables}
\else
  \newcommand\listtablename{List of Tables}
\fi
\ifdefined\figurename
  \renewcommand*\figurename{Figure}
\else
  \newcommand\figurename{Figure}
\fi
\ifdefined\tablename
  \renewcommand*\tablename{Table}
\else
  \newcommand\tablename{Table}
\fi
}
\@ifpackageloaded{float}{}{\usepackage{float}}
\floatstyle{ruled}
\@ifundefined{c@chapter}{\newfloat{codelisting}{h}{lop}}{\newfloat{codelisting}{h}{lop}[chapter]}
\floatname{codelisting}{Listing}
\newcommand*\listoflistings{\listof{codelisting}{List of Listings}}
\makeatother
\makeatletter
\makeatother
\makeatletter
\@ifpackageloaded{caption}{}{\usepackage{caption}}
\@ifpackageloaded{subcaption}{}{\usepackage{subcaption}}
\makeatother
\ifLuaTeX
  \usepackage{selnolig}  % disable illegal ligatures
\fi
\usepackage[style=phys,eprint=true,url=true,backref=true,biblabel=brackets,citestyle
= numeric-comp,sorting = none]{biblatex}
\addbibresource{references.bib}
\usepackage{bookmark}

\IfFileExists{xurl.sty}{\usepackage{xurl}}{} % add URL line breaks if available
\urlstyle{same} % disable monospaced font for URLs
\hypersetup{
  pdftitle={A Lecture on Topological Operators},
  pdfauthor={Kantaro Ohmori},
  colorlinks=true,
  linkcolor={blue},
  filecolor={Maroon},
  citecolor={Blue},
  urlcolor={Blue},
  pdfcreator={LaTeX via pandoc}}

\title{A Lecture on Topological Operators}
\author{Kantaro Ohmori}
\date{2024-01-30}

\begin{document}
\maketitle

\renewcommand*\contentsname{Table of contents}
{
\hypersetup{linkcolor=}
\setcounter{tocdepth}{2}
\tableofcontents
}
\bookmarksetup{startatroot}

\chapter*{What is this}\label{what-is-this}
\addcontentsline{toc}{chapter}{What is this}

\markboth{What is this}{What is this}

\(\vol\) This is a lecture note prepared for two sets of ``intensive
lectures'':\footnote{In Japan, an ``intensive lecture'' is a format of a
  lecture course where a lecturer (usually from another university)
  gives lectures in consecutive days filling 7-9 slots in usually 3
  days.}

\begin{itemize}
\tightlist
\item
  at Tohoku University, Oct.~11-13, 2023, and
\item
  at Yukawa Insititute for Theoretical Physics, Kyoto University,
  Nov.~29-1, 2023.
\end{itemize}

In this lecture I will try to explain the constructions of topological
defects corresponding to generalized symmetries. Due to lack of time and
(more significantly) my understanding, the lecture will focus on bosonic
systems, and the generalization to fermionic systems is left for the
readers/audiences.

\begin{tcolorbox}[enhanced jigsaw, toprule=.15mm, colback=white, rightrule=.15mm, breakable, colframe=quarto-callout-warning-color-frame, title=\textcolor{quarto-callout-warning-color}{\faExclamationTriangle}\hspace{0.5em}{}, bottomrule=.15mm, opacityback=0, colbacktitle=quarto-callout-warning-color!10!white, bottomtitle=1mm, left=2mm, coltitle=black, leftrule=.75mm, arc=.35mm, titlerule=0mm, toptitle=1mm, opacitybacktitle=0.6]

This note is \textbf{under construction}, and there are many missed
equations, figures, explanations, sections, and \emph{references}.

\end{tcolorbox}

\section*{Prerequisite}\label{prerequisite}
\addcontentsline{toc}{section}{Prerequisite}

\markright{Prerequisite}

\begin{itemize}
\tightlist
\item
  Basic knowledge about scalar field theory and (abelian) gauge theory
  in path-integral formalism, and
\item
  Knowledge about renormalization group (RG) flows to understand
  motivations.
\item
  Knowledge about differential form and Stokes's theorem in terms of it.
\end{itemize}

\section*{What is contained and what is
not}\label{what-is-contained-and-what-is-not}
\addcontentsline{toc}{section}{What is contained and what is not}

\markright{What is contained and what is not}

\section*{Other Lectures/Reviews}\label{other-lecturesreviews}
\addcontentsline{toc}{section}{Other Lectures/Reviews}

\markright{Other Lectures/Reviews}

Recently there has been a surge of lecture notes/ review articles on
generalized symmetries. The ones I have noticed are
\autocite{McGreevy:2022oyu,Schafer-Nameki:2023jdn,Gomes:2023ahz,Bhardwaj:2023kri,Luo:2023ive,Shao:2023gho}.
Because this lecture will focus on the fundamental aspects of the topic
and will not connect very well with the existent literature (so sorry
about that), readers/audiences are strongly encouraged to refer to at
least one of them, or something similar.

Also, about conventional symmetries and their anomalies, there are nice
old lectures. The one I would particularly recommend is
\autocite{TachikawaTasi}.

\bookmarksetup{startatroot}

\chapter{Introduction}\label{introduction}

\section{Symmetry}\label{symmetry}

\textbf{Symmetry} plays a crucial role in theoretical physics. In this
lecture, we will discuss its application in \emph{quantum field
theories} (QFTs). A fundamental aspect of symmetry in QFTs is its
preservation along the renormalization group flow. More precisely, when
an ultraviolet (UV) theory \(\mathcal{T}_\text{UV}\) transitions into an
infrared theory \(\mathcal{T}_\text{IR}\), a canonical homomorphism
\(f_\text{RG}\) exists from the UV symmetry group \(G_\text{UV}\) to the
IR symmetry group \(G_\text{IR}\):{[}\^{}SymRG{]}

\begin{tcolorbox}[enhanced jigsaw, toprule=.15mm, colback=white, rightrule=.15mm, breakable, colframe=quarto-callout-note-color-frame, title=\textcolor{quarto-callout-note-color}{\faInfo}\hspace{0.5em}{}, bottomrule=.15mm, opacityback=0, colbacktitle=quarto-callout-note-color!10!white, bottomtitle=1mm, left=2mm, coltitle=black, leftrule=.75mm, arc=.35mm, titlerule=0mm, toptitle=1mm, opacitybacktitle=0.6]

Note that there are five types of callouts, including: \texttt{note},
\texttt{warning}, \texttt{important}, \texttt{tip}, and
\texttt{caution}.

\end{tcolorbox}


\printbibliography


\end{document}
